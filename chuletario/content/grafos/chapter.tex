\chapter{Grafos}

\section{Recorridos}
	\chuletarioimport{dfs.cpp}{}%
	[][][\bigo{V + E}]
	
	\chuletarioimport{bfs.cpp}{}% multi-source : UVa 11101
	[Recorrido en anchura. También resuelve SSSP si el grafo es sin pesos. Multi-source BFS: mete todos los nodos fuentes a una \texttt{queue}.]%
	[][\bigo{V + E}]
	
	\chuletarioimport{floodfill.cpp}{}%
	[Hace floodfill de la componente conexa del vértice (\texttt{r}, \texttt{c}) de color \texttt{c1} coloreándola de color \texttt{c2}.]
\medskip
\subsection{Graph Check}
	\chuletarioimport{graphCheck.cpp}{}%
	[Clasificación de tipos de aristas: tree edge (una arista del DFS spanning tree), back edge (una arista que forma parte de un ciclo) y forward cross edge (una arista de un vértice \texttt{EXPLORED} a un vértice \texttt{VISITED}).]
	
	\chuletarioimport{bipartiteGraphCheck.cpp}{}%
	[Intenta bicolorear un grafo para detectar si es bipartito.]

\medskip
\subsection{Articulation points and bridges en grafos no dirigidos}
	En un grafo no dirigido $G$, un \textit{articulation point} es un vértice que al ser eliminado (junto con sus aristas) desconecta $G$.\\
	Un \textit{bridge} es una arista que al ser eliminada desconecta $G$.

	\chuletarioimport{articulationPointsAndBridges.cpp}{}%
	[Encuentra articulation points and bridges. \texttt{dfs\_num[u]} es el número de iteración en el que se visita $u$ \textit{por primera vez}. \texttt{dfs\_low[u]} es el valor de \texttt{dfs\_num[v]} más bajo de un vértice $v$ alcanzable desde el subárbol de expansión del recorrido DFS que empieza en $u$.\\
	\textbf{Caso especial:} la raíz del árbol DFS (\texttt{dfsRoot}) es un articulation point si y sólo si tiene más de un hijo (\texttt{rootChildren > 1}).]%
	[][\bigo{V + E}]

\section{SSSP}
	\chuletarioimport{dijkstra.cpp}{}%
	[SSSP desde el vértice \texttt{s} en un grafo dirigido con pesos \textit{sin ciclos de coste negativo}. \texttt{adjList[u][j]} es un par (\texttt{w}, \texttt{v}) donde \texttt{w} es el peso de la arista $u \rightarrow v$.]%
	[][\bigo{(V + E) \log V}, probablemente TLE si $V, E > 300K$.]
	
	\chuletarioimport{bellmanFord.cpp}{}%
	[SSSP desde el vértice \texttt{s} en un grafo dirigido con pesos. Devuelve si existe al menos un ciclo de coste negativo.\\
	Si quieres saber los vértices que están en un ciclo de coste negativo, puedes encontrar las SCCs de los vértices $v$ con \texttt{relax[v] = 1}.\\
	Si quieres saber los vértices hasta los que puedes llegar con coste negativo, recorre el grafo desde los vértices $v$ con \texttt{relax[v] = 1}.]%
	[][\bigo{VE}, probablemente TLE si $VE > 10^6$.]

\section{APSP}
	\chuletarioimport{floyd.cpp}{}%
	[APSP en un grafo dirigido con pesos en matriz de adyacencia \texttt{adjMat} de \texttt{N} vértices.\\
	Asegúrate de inicializar \texttt{adjMat[i][i] = 0} y \texttt{adjMat[i][j] = INF} si no existe la arista $i \rightarrow j$ (no uses \texttt{INT\_MAX} por overflow).\\
	La matriz \texttt{pi} sirve para imprimir el camino mínimo: \texttt{pi[i][j]} es el vértice anterior a $j$ en un camino mínimo de $i$ a $j$; $\texttt{i} \rightarrow ... \rightarrow$ \texttt{pi[i][j]} $\rightarrow$ \texttt{j}.]%
	[][\bigo{V^3}, probablemente TLE si $V > 400$]

\section{MST}
	\chuletarioimport{kruskal.cpp}{}%
	[Añade aristas vorazmente al MST hasta que todos los vértices están en el MST.\\
	Ordena \texttt{edgeList}. Devuelve \texttt{INT\_MAX} si el grafo no es conexo.]%
	[][\bigo{E \log E} $\approx$ \bigo{E \log V^2} = \bigo{2E \log V} = \bigo{E \log V}]
	
	Problema del \textit{minimax} (análogamente se define \textit{maximin}): dados dos vértices de un grafo $i$ y $j$, considera todos los caminos entre ellos, y asocia a cada camino un coste: el peso máximo de entre todas las aristas que forman el camino. ¿Cuál es el menor coste de estos caminos? Solución: un camino con coste mínimo es el que une $i$ y $j$ en el MST del grafo.

\section{SCC}
	\chuletarioimport{tarjan.cpp}{}%
	[Calcula las SCCs de un grafo dirigido y las guarda en \texttt{sccs}. \texttt{cur\_scc[v]} $= 1 \iff v$ está en la SCC que se está explorando actualmente, cuyos vértices se guardan en la ``pila'' \texttt{S}. Un vértice $v$ es el ``inicio'' (según el recorrido del DFS) de una SCC $\iff$ \texttt{dfs\_low[v]} $=$ \texttt{dfs\_num[u]}.]%
	[][\bigo{E + V}]
	
	\chuletarioimport{kosaraju.cpp}{}%
	[Calcula el número de SCCs de un grafo dirigido. Dos vértices $u$ y $v$ están en la misma SCC si y sólo si se puede llegar desde $u$ a $v$ en el grafo original y desde $v$ a $u$ en el grafo traspuesto mediante un DFS. Para contar las SCC bien se inicia el recorrido en el grafo traspuesto desde los vértices que van ``antes'' en el grafo original, por lo que se calcula un toposort antes (no es un toposort estricto pues el grafo podría ser cíclico).]%
	[][\bigo{2(E + V)}]

\section{TopoSort}

	Véase DFS.

	\chuletarioimport{kahn.cpp}{}%
	[Calcula un toposort de un grafo dirigido sin pesos. Devuelve \texttt{true} si el grafo es cíclico.]% Ver CF 825E
	[][\bigo{V + E}]

\section{MaxFlow}
	\chuletarioimport{edmondsKarp.cpp}{}%
	[Calcula el flujo máximo en una red de flujo con un único source node \texttt{s} y único sink node \texttt{t} implementando el método de Ford Fulkerson: usa BFS para encontrar un augmenting path. La matriz \texttt{res} contiene las capacidades residuales y se inicializa con las capacidades de las aristas. Pon en \texttt{adjList} las aristas en ambos sentidos.]%
	[][\bigo{VE^2}]

\section{Grafos eulerianos}

	Un grafo no dirigido
	\begin{itemize}
		\item tiene un \textbf{ciclo euleriano} si y sólo si todos los vértices tienen grado par y todos los vértices con grados no nulo pertenecen a la misma componente conexa.
		\item tiene un \textbf{ciclo euleriano} si y sólo si se puede descomponer en edge-disjoint cycles y todos los vértices con grado no nulo pertenecen a la misma componente conexa.
		\item tiene un \textbf{camino euleriano} si y sólo si tiene exactamente cero o dos vértices con grado impar y los vértices con grado no nulo pertenecen a la misma componente conexa.
	\end{itemize}

	Un grafo dirigido
	\begin{itemize}
		\item tiene un \textbf{ciclo euleriano} si y sólo si cada vértice tiene \texttt{inDeg} = \texttt{outDeg} y todos los vértices con grado no nulo pertenecen a la misma componente fuertemente conexa.
		\item tiene un \textbf{ciclo euleriano} si y sólo si se puede descomponer en edge-disjoint cycles y todos los vértices con grado no nulo pertenecen a la misma componente conexa.
		\item tiene un \textbf{camino euleriano} si y sólo si tiene a lo sumo un vértice con \texttt{outDeg - inDeg} $= 1$, a lo sumo un vértice con \texttt{inDeg - outDeg} $= 1$, los demás vértices tienen \texttt{inDeg} $=$ \texttt{outDeg}, y todos los vértices con grado no nulo pertenecen a la misma componente conexa del grafo no dirigido subyacente.
	\end{itemize}

	\chuletarioimport{eulerTour.cpp}{}%

\section{Matchings}
	\begin{description} 
		\item[Matching] Un matching (o independent edge set) de un grafo $G$ es un conjunto de aristas que no tienen vértices en común.
		\item[Augmenting path] Dado un matching $M$ en un grafo $G$, un augmenting path (camino de $M$-aumento) es un camino que empieza y acaba en unmatched vertices y cuyas aristas alternan entre aristas que están y no esan en $M$.
		\item[Lema de Berge] Un matching $M$ es máximo (contiene la mayor cantidad de aristas) si y sólo si no tiene un augmenting path.
		\item[Vertex cover] Un vertex cover es un conjunto de vértices tales que cada arista del grafo es incidente a al menos un vértice del conjunto.
		\item[Conjunto independiente de vértices (independent set)] Es un conjunto de vértices tales que no existe ninguna arista entre ellos.	
	\end{description}

	\subsection{Grafos bipartitos}
		\begin{description}
			\item[Teorema de König] En un grafo bipartito, el número de aristas en un matching máximo es igual al número de vértices en un recubrimiento mínimo de vértices (i.e. MaxCBM = MinVC).
			\item[MaxIS + MaxCBM = V] en grafos bipartitos.
		\end{description}
	
		\chuletarioimport{bergeMcbm.cpp}{}%
		[Calcula un matching máximo en un grafo bipartito usando el lema de Berge. \texttt{M} es el tamaño de la partición izquierda del grafo, \texttt{N} la de la derecha. \texttt{adjList} es un grafo dirigido \textbf{de tamaño \texttt{M}} con las aristas dirigidas desde la partición izquierda a la derecha. \texttt{match} indica con qué vértices de la partición izquierda están emparejados los vértices de la partición derecha; tiene tamaño \texttt{N + M} pero sólo se indexa con los vértices de la partición derecha.]%
		[][\bigo{VE}]
